\documentclass{article}
\usepackage{graphicx} % Required for inserting images

\usepackage{float} 
\usepackage{color}
\usepackage[a4paper, total={6in, 8in},left=2cm,right=2cm,
top=2cm,bottom=2cm,bindingoffset=0cm]{geometry}
\usepackage[utf8]{inputenc}
\usepackage[english, russian]{babel}
\usepackage{amsmath}
\usepackage{amsfonts}

\usepackage{xfrac}

\title{Правила семантики}
\author{sophy-spb }
\date{March 2025}

\begin{document}



\section{Правила семантики для DSL}

\subsection{Общие свойства}
\begin{itemize}
    \item Следующие свойства допустимы для всех типов объектов:
    \begin{itemize}
        \item \texttt{color} (строка): Цвет объекта. Допустимые значения: \texttt{NONE, RED, GREEN, BLUE, BLACK, WHITE, YELLOW, PURPLE}.
        \item \texttt{text} (строка): Текст, связанный с объектом.
        \item \texttt{border} (строка): Ширина границы объекта.
        \item \texttt{x} (целое число): Координата центра X объекта.
        \item \texttt{y} (целое число): Координата центра Y объекта.
        \item \texttt{size\_text} (число, $> 0$): Размер текста.
    \end{itemize}
\end{itemize}

\subsection{Уникальные свойства для каждого типа объекта}
\begin{itemize}
    \item \textbf{Круг (\texttt{circle})}:
    \begin{itemize}
        \item \texttt{radius} (число, $> 0$): Радиус круга.
    \end{itemize}
    \item \textbf{Прямоугольник (\texttt{rectangle})}:
    \begin{itemize}
        \item \texttt{size\_A} (число, $> 0$): Длина первой стороны.
        \item \texttt{size\_B} (число, $> 0$): Длина второй стороны.
    \end{itemize}
    \item \textbf{Ромб (\texttt{diamond})}:
    \begin{itemize}
        \item \texttt{size\_A} (число, $> 0$): Длина первой диагонали.
        \item \texttt{size\_B} (число, $>0$): Длина второй диагонали.
        \item \texttt{angle} (число, $0 < \text{значение} < 360$): Угол между сторонами ромба.
    \end{itemize}
    \item \textbf{Облако точек (\texttt{dot\_cloud})}:
    \begin{itemize}
        \item \texttt{grid} (логическое): Наличие сетки.
    \end{itemize}
\end{itemize}

\subsection{Обязательные свойства}
\begin{itemize}
    \item Для точек в облаке точек (\texttt{dot\_cloud}):
    \begin{itemize}
        \item Каждая точка должна иметь обязательные свойства \texttt{x} и \texttt{y}.
    \end{itemize}
\end{itemize}

\subsection{Проверка значений свойств}
\begin{itemize}
    \item Значения свойств должны соответствовать следующим ограничениям:
    \begin{itemize}
        \item Числовые свойства (\texttt{radius}, \texttt{size\_A}, \texttt{size\_B}, \texttt{angle}, \texttt{x}, \texttt{y}, \texttt{size\_text}, \texttt{border}):
        \begin{itemize}
            \item Должны быть числами.
            \item Должны удовлетворять указанным ограничениям (например, \texttt{radius} $> 0$, \texttt{angle} от 0 до 360).
        \end{itemize}
        \item Строковые свойства (\texttt{color}, \texttt{text}):
        \begin{itemize}
            \item Должны быть строками.
            \item Для \texttt{color} допустимы только предопределённые значения.
        \end{itemize}
        \item Логические свойства (\texttt{grid}):
        \begin{itemize}
            \item Должны быть \texttt{true} или \texttt{false}.
        \end{itemize}
    \end{itemize}
\end{itemize}

\subsection{Проверка уникальности идентификаторов объектов}
\begin{itemize}
    \item Идентификатор объекта (\texttt{ID}) должен быть уникальным в пределах своей области видимости.
    \item Области видимости создаются для:
    \begin{itemize}
        \item Основной программы
        \item Каждого графа (\texttt{graph})
        \item Каждого облака точек (\texttt{dot\_cloud})
    \end{itemize}
\end{itemize}

\subsection{Проверка допустимости свойств}
\begin{itemize}
    \item Каждое свойство объекта должно быть допустимым для его типа.
    %\item Если свойство недопустимо для данного типа объекта, выдается ошибка.
\end{itemize}

\subsection{Проверка отношений (\texttt{relation})}
\begin{itemize}
    \item Оба объекта, указанные в отношении, должны быть объявлены.
    \item Свойства отношений должны быть допустимыми (см. общие свойства).
    \item Связь не может быть создана с графом (\texttt{graph}) или облаком точек (\texttt{dot\_cloud}).
\end{itemize}

\subsection{Проверка дублирования связей в графе}
\begin{itemize}
    \item Внутри графа (\texttt{graph}) не должно быть двух связей между одними и теми же объектами.
\end{itemize}

\subsection{Проверка облака точек (\texttt{dot\_cloud})}
\begin{itemize}
    \item Каждая точка в облаке точек должна иметь свойства \texttt{x} и \texttt{y}.
    \item Свойства облака точек должны быть допустимыми.
\end{itemize}

\subsection{Проверка заметок (\texttt{note})}
\begin{itemize}
    \item Свойства заметок должны быть допустимыми (см. общие свойства).
\end{itemize}

\subsection{Проверка графа (\texttt{graph})}
\begin{itemize}
    \item Свойства графа должны быть допустимыми (см. общие свойства).
    \item Все объекты и отношения внутри графа должны быть корректными.
\end{itemize}

\subsection{Обработка ошибок}
\begin{itemize}
    \item При обнаружении семантической ошибки анализатор должен:
    \begin{itemize}
        \item Вывести сообщение об ошибке с указанием типа ошибки и номера строки.
        \item Завершить выполнение программы.
    \end{itemize}
\end{itemize}

%\section{Примеры ошибок}

%\subsection{Недопустимое свойство}
%\begin{verbatim}
%circle myCircle { radius = 10; angle = 45 }  # Ошибка: angle недопустим для circle
%\end{verbatim}
%\begin{itemize}
%    \item \textbf{Сообщение об ошибке}: \texttt{Недопустимое свойство angle для фигуры circle в строке 2}.
%\end{itemize}

%\subsection{Отсутствие обязательных свойств}
%\begin{verbatim}
%dot_cloud myCloud {
%    circle point1 { x = 10 }  # Ошибка: отсутствует y
%}
%\end{verbatim}
%\begin{itemize}
%    \item \textbf{Сообщение об ошибке}: \texttt{Для точки в облаке точек обязательны свойства x и y. Строка 3}.
%\end{itemize}

%\subsection{Дублирование связей в графе}
%\begin{verbatim}
%graph MyGraph {
%    circle A { x = 10; y = 20 }
%    circle B { x = 30; y = 40 }
%    A -> B
%    B -> A  # Ошибка: объекты уже соединены в графе
%}
%\end{verbatim}
%\begin{itemize}
 %   \item \textbf{Сообщение об ошибке}: \texttt{Объекты A и B уже соединены в графе. Строка 6}.
%\end{itemize}

%\subsection{Необъявленный объект}
%\begin{verbatim}
%A -> B  # Ошибка: объект B не объявлен
%circle B { x = 30; y = 40 }
%\end{verbatim}
%\begin{itemize}
%    \item \textbf{Сообщение об ошибке}: \texttt{Объект B не объявлен. Строка 2}.
%\end{itemize}

%\subsection{Связь с графом или облаком точек}
%\begin{verbatim}
%graph MyGraph { ... }
%circle A { ... }
%MyGraph -> A  # Ошибка: связь не может быть создана с графом
%\end{verbatim}
%\begin{itemize}
%    \item \textbf{Сообщение об ошибке}: \texttt{Связь не может быть создана с графом или облаком точек. Строка 3}.
%\end{itemize}

\end{document}
